%Generales 
\documentclass[letterpaper,12pt]{article} % Documento en dos columnas, tamaño carta
\usepackage[spanish]{babel}       % Traduce capítulos, fechas, etc. al español
\usepackage[utf8]{inputenc}       % Permite usar acentos directamente
\usepackage[T1]{fontenc}          % Codificación para que se vean bien los caracteres en PDF
\usepackage{lmodern}              % Usa fuentes escalables compatibles con microtype
\usepackage{geometry}             % Control de márgenes y tamaño de página
\usepackage{fancyhdr}            % Encabezados y pies de página personalizados

%Notación 

\usepackage{amsmath}  % Ecuaciones, símbolos y fuentes matemáticas
\usepackage{amssymb}
\usepackage{amsfonts}
\usepackage{siunitx}                     % Escritura coherente de unidades del SI y números

%gráficos 

\usepackage{graphicx}           % Insertar y escalar imágenes
\usepackage{subcaption}         % Subfiguras dentro de una figura
\usepackage{booktabs}           % Tablas más profesionales
\usepackage{longtable}          % Tablas que ocupan varias páginas
\usepackage{tabularx}           % Tablas con ancho ajustable automáticamente


% --- COLORES Y LISTAS PERSONALIZADAS ---
\usepackage{xcolor}             % Uso de colores en texto y figuras
\usepackage{enumitem}           % Listas personalizadas (espaciado, numeración)

% --- HIPERVÍNCULOS Y NAVEGACIÓN ---
\usepackage{hyperref}           % Enlaces ciclables en PDF (índice, URLs, referencias)

% --- DIAGRAMAS TÉCNICOS ---
\usepackage{tikz}                         % Gráficos vectoriales (bloques, flujos, redes)
\usepackage{forest}                      % Árboles, jerarquías (estructuras)

% --- ALGORITMOS Y CÓDIGO FUENTE ---
\usepackage{algorithm2e}    % Pseudocódigo paso a paso (algoritmos)
\usepackage{listings}       % Mostrar código fuente con formato

\geometry{letterpaper, margin=1in} % Márgenes de 1 pul
\title{Ayahuik 1: Guía de Misión}
\author{CUAUHTÉMOC IPN}


\pagestyle{fancy} 
\setlength{\headheight}{14.5pt}
\fancyhf{}
\fancyhead[C]{Cuauhtemoc IPN}
\fancyhead[R]{AYAHUIK 1}
\fancyhead[L]{Guia de Misión}
\fancyfoot[R]{\thepage}

\graphicspath{{Imagenes/}}

\begin{document}

\section{Antecedentes}

    \subsection{Antecedentes Históricos}

    \subsection{Antecedentes IPN}

    \subsection{Antecedentes Cuauhtémoc IPN}

\section{Descripción general de la misión}

    \subsection{Participantes de la misión}

    \subsection{CONOPS}

    \subsection{Descripción de la carga util}

    \subsection{Descripción del proceso de diseño y construcción}

    \subsection{Descripción del lanzamiento y recuperación}

\section{Objetivos y criterios de éxito}

    \subsection{Objetivos generales}

    \subsection{Objetivos específicos}
    
    \subsection{Criterios de éxito}

\section{Resultados esperados}

    \subsection{Resultados técnicos}

    \subsection{Resultados de la misión}

    \subsection{Resultados de la carga útil}
\newpage

\section{Organización del equipo}

    El equipo Cuauhtémoc IPN está organizado en diferentes subsecciones para asegurar el funcionamiento del equipo y una gestión 
    eficiente de las misión, todas estas serán coordinadas por los líderes de misión 
    y estos a su ves por los capitanes del equipo. 
    \begin{figure}[!h]
      \centerline{\includegraphics[width=.5\textwidth]{ORG-AYA1-GM-V3.png}}
      \caption{Organigrama Ayahuik 1}
      \label{1}
    \end{figure}
    
    Para la misión Ayahuik 1, el equipo Cuauhtémoc IPN contara con las 2 capitanas del equipo, 
    las cuales se encargaran de coordinar las actividades generales del equipo asi como sus misiones activas, 
    estas son:

    \begin{itemize}
        \item \textbf{Capitana:} Sofia Rojas
        \item \textbf{Subcapitana:} Tania Olmos
    
    \end{itemize}

    De esto se derivara el asesor técnico el cual se encargara de todo el asesoramiento tanto para el funcionamiento del equipo
    como para la correcta realización de la misión sin que este tenga intervención directa, el cual es:

    \begin{itemize}
        \item \textbf{Asesor técnico:} Hector Diaz

    \end{itemize}

    El liderato de la misión AYAHUIK 1 estará a cargo de 3 co-líderes,
    los cuales se encargaran de coordinar el correcto funcionamiento de Cada
    una de las subsecciones del equipo, estos son:

    \begin{itemize}
        \item \textbf{Líder administrativo:} Luis Romero
        \item \textbf{Líder técnico:} Andrea Garcia
        \item \textbf{Líder de convergencia:} Brandon Garcia
    
    \end{itemize}

    De los cuales se derivaran las subsecciones del equipo, las cuales son aeroestructuras, EPS (Electrical Power System), GS (Ground Station) y CDH (Communication and Data Handling).
\end{document}
